\documentclass[a4paper,12pt]{article}
\usepackage[francais]{babel}
\usepackage[T1]{fontenc}
\usepackage[utf8]{inputenc}
\selectlanguage{francais}
\title{Ihy Project}
\author{
Ihy Group : \\
Teddy Wagnac : \textit{wagnac\_t}, Stéphane Sezer : \textit{sezer\_s},\\
Nicolas Geniteau : \textit{genite\_n}, Xavier Deguillard : \textit{deguil\_x}
}
\date{24 octobre 2008}

\begin{document}

\maketitle

\section*{Qui sommes-nous ?}
L'équipe du Ihy project est composée de deux élèves de A2, un élève de B1, ainsi
que d'un élève de B2.  Ces derniers se nomment	Teddy  Wagnac,	Stéphane  Sezer,
Nicolas Geniteau et Xavier Deguillard. Le chef de projet sera Stéphane Sezer de
A2.  L'équipe sera appelée	le	Ihy  Group,  en  référence	au	nom  du  projet.

\section*{Présentation du projet}
Ihy, quel nom étrange pour un projet. Dans, la lignée de Donald Knuth et de ses
"clowneries", l'équipe a choisi un nom imprononçable pour la représenter.	Mais
ce nom a une signification.  Il  s'agit  en  effet	de	celui  d'un  jeune	dieu
égyptien.  On le  prénommait  "le  musicien"  ou  "le  joueur  de  sistre",  nom
particulièrement adapté pour notre projet.\\ Partis d'une idée de création	d'un
lecteur audio avec fonctionnalités avancées, nous nous sommes rendu  compte  que
ce projet n'avait que peu,	voire  aucun  intérêt  algorithmique.	Au	fil  des
réflexions, nous nous somme donc orientés vers l'élaboration d'un  codec  audio.
En effet, les intérêts de ce type de travail sont  clairement  plus  élevés  que
ceux d'un "simple" lecteur audio.

\section*{Intérêts du projet}
L'étendue d'un travail sur le traitement du signal est assez vaste. Elle englobe
plusieurs  domaines  mathématiques	tels  que  les	 séries   de   Fourier;   et
informatiques, comme le traitement temps réel,	sans  parler,  bien  entendu  du
travail à effectuer sur les algorithmes.

\section*{Ce que nous comptons faire}
Notre travail se découpera en deux parties principales.   La  première	sera  le
"cœur"	du	projet	à  proprement  parler,	c'est  à  dire	les  algorithmes  de
compression/décompression du flux audio, et la seconde sera tout ce qui  servira
de liant à ces derniers, c'est à dire l'interface  graphique  qui  permettra  de
lire (décompression) des fichiers audio au format Ihy ainsi que de	produire  ce
type de fichiers à partir de sources non  compressées  (compression).\\  Sachant
que nous devons utiliser du C et du OCaml pour ce  projet,	nous  avons  décidé,
grosso modo, de faire la première partie en OCaml, et la seconde en C.\\ Le type
de compression en lui même reste à définir.  Nous avons tout de  même  plusieurs
voies possible qui s'offrent à nous.  Tout d'abord,  il  existe  deux  types  de
compression audio.	La première dite sans  perte,  qui	permet	de	restituer  à
l'identique le signal  original  à	partir	du	flux  compressé;  ainsi  que  la
compression dite avec perte, qui offre des taux plus élevés  car  elle	supprime
des informations du flux original.	A priori, nous allons nous orienter vers  de
la compression avec perte, pour ses performances, ainsi  que  l'intérêt  qu'elle
porte au niveau de la compréhension de la perception sonore humaine.\\	Ensuite,
il existe plusieurs manières d'aborder la compression. Celle qui nous intéresse
ici est la compression par	ondelettes.   Cette  technique	sert  normalement  à
traiter des images, mais des travaux sont en cours pour éventuellement l'adapter
au signal audio.  Il est également envisageable de mêler  cette  dernière  à  la
compression par fractales, encore une  fois  utilisée  essentiellement	dans  le
traitement des images.\\ Une autre décision que nous  devrons  prendre	sera  de
limiter ou non le type de données que nous	allons	traiter.   Par	exemple,  il
paraît clair que le traitement de la voix,	et	plus  particulièrement	dans  le
cadre de la téléphonie ne se fera pas de la même façon que	le	traitement	d'un
enregistrement musical.\\ De plus  amples  informations  sur  nos  choix  seront
apportées dans le cahier des charges qui sera remis plus  tard	dans  l'année.\\
Notre  projet  prend  ainsi  clairement  une  approche	orientée  recherche   et
innovation, et les résultats, s'ils sont à la hauteurs de nos attentes,  peuvent
se révéler très intéressants.

\section*{Ajout au synopsis original}
Dans la pratique, notre programme traitera uniquement les  fichiers  PCM  et  si
nous avons le temps et la possibilité, il pourra compresser directement les  CDs
audio.	Le PCM est un fichier numérique codant directement le signal  analogique
d'origine.	Ce format est sans perte et  est  donc	parfaitement  adapté  à  nos
besoins.  A partir de cela, notre programme  compressera  les  données	dans  le
format ihy, ces données seront vraisemblablement compressés avec pertes.   Après
cette phase de compression, l'utilisateur pourra au choix télécharger  les	tags
musicaux ou les ajouter lui même.  Enfin il pourra lire  la  musique  au  format
ihy.\\ En ce qui concerne le travail à fournir,  nous  allons  normalement	nous
orienter vers la compression par ondelettes qui représente	en	soit  une  belle
prouesse si nous l'implantons  correctement  (aucun  codec	audio  ne  l'utilise
actuellement), puis en seconde passe, un  algorithme  de  compression  des	plus
classiques, j'ai nommé Huffman.  Bien  sûr	avant  que	tout  ceci	puisse	être
utilisable il faut que le PCM soit décodé pour pouvoir le manipuler comme on  le
souhaite. Une autre partie des plus importante est la conception de l'interface
graphique qui possèdera un	lecteur  audio	des  plus  classiques  ainsi  qu'une
interface pour compresser les fichiers PCM vers le format  ihy.\\  Le  découpage
des taches se fera comme suit : Une personne sera  chargée	de	l'interface,  et
donc de tout le lecteur, deux personnes travailleront sur l'implémentation de la
compression par ondelettes, aussi bien au niveau de l'écriture des	algorithmes,
que sur le travail de recherche d'information et de documentation qui  précèdera
l'implémentation. En effet, une bonne connaissance des séries de Fourier, ainsi
que de l'outil mathématique  qu'est  la  transformée  de  Fourier  s'avère	être
indispensable pour maitriser les ondelettes.  Malheureusement  nous  ne  verrons
que ces derniers en fin d'année en mathématiques.  Nous devrons  donc  anticiper
sur une partie du programme.  Enfin, une  quatrième  personne  se  chargera  des
algorithmes auxiliaires (comme par exemple Huffman) ainsi que de l'interface  du
codec avec le "monde exterieur".   Le  codec  ihy  devra  pouvoir  être  utilisé
indépendamment du lecteur qui lui est associé au sein du projet. Pour cela nous
devrons séparer très clairement les deux parties du projet énoncée précédemment,
c'est à dire le lecteur et le codec.\\ Cette organisation des tâches permet  une
bonne  souplesse  au  fil  de  l'année	 et   laisse   libre   à   d'éventuelles
réorganisations si certains membres s'avéraient  éprouver  des	difficultés  sur
leur domaine de travail.

\end{document}
