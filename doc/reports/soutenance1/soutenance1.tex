\documentclass[a4paper,12pt]{article}
\usepackage[francais]{babel}
\usepackage[T1]{fontenc}
\usepackage[utf8]{inputenc}
\usepackage{pslatex}
\usepackage{url}
\usepackage{graphicx}
\usepackage{lscape}
\selectlanguage{francais}


\title{Rapport de Soutenance 1}
\author{
Ihy Group : \\
deguil\_x (Xavier Deguillard)\\
genite\_n (Nicolas Geniteau)\\
sezer\_s (Stephane Sezer)\\
wagnac\_t (Teddy Wagnac)
}
\date{10 f\'evrier 2009}

\begin{document}

\maketitle

\newpage

\section*{Introduction}
Nous avons decidé cette année de nous atteler à la réalisation d'un codec audio.
L'orientation principale des projets de spé de cette année étant la recherche et
le développement, nous avons choisi de nous tourner vers les ondelettes qui
seront notre base de travail. En effet, les ondelettes étant un outil
mathématique relativement récent (elles datent du XXème siècle), elles ne sont
pas encore largement utilisées. Il existe actuellement certains algorithmes de
compression vidéo, par exemple Dirac, leur implémentation par la BBC, mais rien
de probant pour le traitement du signal audio.\\
Notre nouveau codec révolutionnaire (du moins nous l'espérons) se nomme donc le
ihy, pour deux raisons. Premièrement parce qu'"ihy" est le nom d'un dieu
egyptien de la musique, et deuxièmement parce que ce mot est compliqué à
prononcer.\\
Ceci étant dit, entrons dans le vif du sujet. Nous présenterons dans ce rapport
le travail accompli depuis la validation du cahier des charges, les problèmes
rencontrés et leurs solutions, ainsi que le travail à accomplir pour la
prochaine soutenance.

\newpage

\tableofcontents

\newpage

\section{Travail accompli}

	\subsection{Spécifications du format ihy}

	\subsection{Ondelette de Haar}

	\subsection{Interfaçage Caml/C}

	\subsection{Algorithme de Huffman}

	\subsection{Premières étapes de threading}

		\subsubsection{Problèmes rencontrés}

	\subsection{Interface graphique en GTK}

		\subsubsection{Contrôles utilisateur}

		\subsubsection{Lecture de fichiers ihy}

\section{Tâches prévues}

	\subsection{Ajout du type half}

	\subsection{Utilisation d'autres ondelettes}

	\subsection{Threading total de la compression/décompression}

	\subsection{Amélioration de l'UI}

\newpage

\section*{Conclusion}
Blabla il fait beau ...

\end{document}
